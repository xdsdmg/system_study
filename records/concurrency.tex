\documentclass[letterpaper,11pt]{article}
\usepackage{CJKutf8}
\usepackage{bm}
\usepackage{geometry}

\geometry{a4paper,left=2cm,right=2cm,top=1cm,bottom=1cm}

\begin{document}
\begin{CJK}{UTF8}{gbsn}

\title{并发}
\author{张驰}
\maketitle

\section{序言}

序言章节给出了一个很形象的并发例子,假如桌子上有很多桃子,每个人可以取一个,这时有两种方式:(1)大家一起拿,先想好要取哪一个再去取,有可能想好的桃子先被其他人取走了;(2)大家排成一对,按顺序取桃子,但这种方式会比较慢。

\section{介绍}

每个线程都会有自己的 stack。

\begin{itemize}
\item Critical section
\item Race condition
\item Indeterminate
\item Mutual exclusion
\end{itemize}

\section {Interlude: Thread API}

\begin{itemize}
\item Keep it simple
\item 不要返回指向线程 stack 空间的指针。
\end{itemize}



\end{CJK}
\end{document}

